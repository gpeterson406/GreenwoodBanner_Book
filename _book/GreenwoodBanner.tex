\documentclass[]{book}
\usepackage{lmodern}
\usepackage{amssymb,amsmath}
\usepackage{ifxetex,ifluatex}
\usepackage{fixltx2e} % provides \textsubscript
\ifnum 0\ifxetex 1\fi\ifluatex 1\fi=0 % if pdftex
  \usepackage[T1]{fontenc}
  \usepackage[utf8]{inputenc}
\else % if luatex or xelatex
  \ifxetex
    \usepackage{mathspec}
  \else
    \usepackage{fontspec}
  \fi
  \defaultfontfeatures{Ligatures=TeX,Scale=MatchLowercase}
\fi
% use upquote if available, for straight quotes in verbatim environments
\IfFileExists{upquote.sty}{\usepackage{upquote}}{}
% use microtype if available
\IfFileExists{microtype.sty}{%
\usepackage{microtype}
\UseMicrotypeSet[protrusion]{basicmath} % disable protrusion for tt fonts
}{}
\usepackage[margin=1in]{geometry}
\usepackage{hyperref}
\hypersetup{unicode=true,
            pdftitle={A Second Semester Statistics Course with R},
            pdfauthor={Mark Greenwood and Katherine Banner},
            pdfborder={0 0 0},
            breaklinks=true}
\urlstyle{same}  % don't use monospace font for urls
\usepackage{natbib}
\bibliographystyle{apalike}
\usepackage{color}
\usepackage{fancyvrb}
\newcommand{\VerbBar}{|}
\newcommand{\VERB}{\Verb[commandchars=\\\{\}]}
\DefineVerbatimEnvironment{Highlighting}{Verbatim}{commandchars=\\\{\}}
% Add ',fontsize=\small' for more characters per line
\usepackage{framed}
\definecolor{shadecolor}{RGB}{248,248,248}
\newenvironment{Shaded}{\begin{snugshade}}{\end{snugshade}}
\newcommand{\KeywordTok}[1]{\textcolor[rgb]{0.13,0.29,0.53}{\textbf{#1}}}
\newcommand{\DataTypeTok}[1]{\textcolor[rgb]{0.13,0.29,0.53}{#1}}
\newcommand{\DecValTok}[1]{\textcolor[rgb]{0.00,0.00,0.81}{#1}}
\newcommand{\BaseNTok}[1]{\textcolor[rgb]{0.00,0.00,0.81}{#1}}
\newcommand{\FloatTok}[1]{\textcolor[rgb]{0.00,0.00,0.81}{#1}}
\newcommand{\ConstantTok}[1]{\textcolor[rgb]{0.00,0.00,0.00}{#1}}
\newcommand{\CharTok}[1]{\textcolor[rgb]{0.31,0.60,0.02}{#1}}
\newcommand{\SpecialCharTok}[1]{\textcolor[rgb]{0.00,0.00,0.00}{#1}}
\newcommand{\StringTok}[1]{\textcolor[rgb]{0.31,0.60,0.02}{#1}}
\newcommand{\VerbatimStringTok}[1]{\textcolor[rgb]{0.31,0.60,0.02}{#1}}
\newcommand{\SpecialStringTok}[1]{\textcolor[rgb]{0.31,0.60,0.02}{#1}}
\newcommand{\ImportTok}[1]{#1}
\newcommand{\CommentTok}[1]{\textcolor[rgb]{0.56,0.35,0.01}{\textit{#1}}}
\newcommand{\DocumentationTok}[1]{\textcolor[rgb]{0.56,0.35,0.01}{\textbf{\textit{#1}}}}
\newcommand{\AnnotationTok}[1]{\textcolor[rgb]{0.56,0.35,0.01}{\textbf{\textit{#1}}}}
\newcommand{\CommentVarTok}[1]{\textcolor[rgb]{0.56,0.35,0.01}{\textbf{\textit{#1}}}}
\newcommand{\OtherTok}[1]{\textcolor[rgb]{0.56,0.35,0.01}{#1}}
\newcommand{\FunctionTok}[1]{\textcolor[rgb]{0.00,0.00,0.00}{#1}}
\newcommand{\VariableTok}[1]{\textcolor[rgb]{0.00,0.00,0.00}{#1}}
\newcommand{\ControlFlowTok}[1]{\textcolor[rgb]{0.13,0.29,0.53}{\textbf{#1}}}
\newcommand{\OperatorTok}[1]{\textcolor[rgb]{0.81,0.36,0.00}{\textbf{#1}}}
\newcommand{\BuiltInTok}[1]{#1}
\newcommand{\ExtensionTok}[1]{#1}
\newcommand{\PreprocessorTok}[1]{\textcolor[rgb]{0.56,0.35,0.01}{\textit{#1}}}
\newcommand{\AttributeTok}[1]{\textcolor[rgb]{0.77,0.63,0.00}{#1}}
\newcommand{\RegionMarkerTok}[1]{#1}
\newcommand{\InformationTok}[1]{\textcolor[rgb]{0.56,0.35,0.01}{\textbf{\textit{#1}}}}
\newcommand{\WarningTok}[1]{\textcolor[rgb]{0.56,0.35,0.01}{\textbf{\textit{#1}}}}
\newcommand{\AlertTok}[1]{\textcolor[rgb]{0.94,0.16,0.16}{#1}}
\newcommand{\ErrorTok}[1]{\textcolor[rgb]{0.64,0.00,0.00}{\textbf{#1}}}
\newcommand{\NormalTok}[1]{#1}
\usepackage{longtable,booktabs}
\usepackage{graphicx,grffile}
\makeatletter
\def\maxwidth{\ifdim\Gin@nat@width>\linewidth\linewidth\else\Gin@nat@width\fi}
\def\maxheight{\ifdim\Gin@nat@height>\textheight\textheight\else\Gin@nat@height\fi}
\makeatother
% Scale images if necessary, so that they will not overflow the page
% margins by default, and it is still possible to overwrite the defaults
% using explicit options in \includegraphics[width, height, ...]{}
\setkeys{Gin}{width=\maxwidth,height=\maxheight,keepaspectratio}
\IfFileExists{parskip.sty}{%
\usepackage{parskip}
}{% else
\setlength{\parindent}{0pt}
\setlength{\parskip}{6pt plus 2pt minus 1pt}
}
\setlength{\emergencystretch}{3em}  % prevent overfull lines
\providecommand{\tightlist}{%
  \setlength{\itemsep}{0pt}\setlength{\parskip}{0pt}}
\setcounter{secnumdepth}{5}
% Redefines (sub)paragraphs to behave more like sections
\ifx\paragraph\undefined\else
\let\oldparagraph\paragraph
\renewcommand{\paragraph}[1]{\oldparagraph{#1}\mbox{}}
\fi
\ifx\subparagraph\undefined\else
\let\oldsubparagraph\subparagraph
\renewcommand{\subparagraph}[1]{\oldsubparagraph{#1}\mbox{}}
\fi

%%% Use protect on footnotes to avoid problems with footnotes in titles
\let\rmarkdownfootnote\footnote%
\def\footnote{\protect\rmarkdownfootnote}

%%% Change title format to be more compact
\usepackage{titling}

% Create subtitle command for use in maketitle
\newcommand{\subtitle}[1]{
  \posttitle{
    \begin{center}\large#1\end{center}
    }
}

\setlength{\droptitle}{-2em}
  \title{A Second Semester Statistics Course with R}
  \pretitle{\vspace{\droptitle}\centering\huge}
  \posttitle{\par}
  \author{Mark Greenwood and Katherine Banner}
  \preauthor{\centering\large\emph}
  \postauthor{\par}
  \predate{\centering\large\emph}
  \postdate{\par}
  \date{2017-06-26}

\usepackage{booktabs}
\usepackage{amsmath}
\usepackage{color}

\definecolor{purple}{RGB}{76,0,153}

\begin{document}
\maketitle

{
\setcounter{tocdepth}{1}
\tableofcontents
}
\chapter*{Acknowledgments}\label{acknowledgments}
\addcontentsline{toc}{chapter}{Acknowledgments}

\chapter{Placeholder}\label{placeholder}

\chapter{Placeholder}\label{placeholder-1}

\chapter{One-Way ANOVA}\label{chapter3}

\section{Situation}\label{section3-1}

In Chapter \ref{chapter2}, tools for comparing the means of two groups
were considered. More generally, these methods are used for a
quantitative response and a categorical explanatory variable (group)
which had two and only two levels. The full prisoner rating data set
actually contained three groups (Figure \ref{fig:Figure3-1} with
\emph{Beautiful}, \emph{Average}, and \emph{Unattractive} rated pictures
randomly assigned to the subjects for sentence ratings. In a situation
with more than two groups, we have two choices. First, we could rely on
our two group comparisons, performing tests for every possible pair
(\emph{Beautiful} vs\emph{Average}, \emph{Beautiful} vs
\emph{Unattractive}, and \emph{Average} vs \emph{Unattractive}). We
spent Chapter \ref{chapter2} doing inferences for differences between
\emph{Average} and \emph{Unattractive}. The other two comparisons would
lead us to initially end up with three p-values and no direct answer
about our initial question of interest -- is there some overall
difference in the average sentences provided across the groups? In this
chapter, we will learn a new method, called \textbf{\emph{Analysis of
Variance}}, or \textbf{\emph{One-Way ANOVA}} since there is just
one\footnote{In Chapter \ref{chapter4}, methods are discussed for when
  there are two categorical explanatory variables that is called the
  Two-Way ANOVA and related ANOVA tests are used in Chapter
  \ref{chapter8} for working with extensions of these models.} grouping
variable. After we perform our One-Way ANOVA test for overall evidence
of a difference, we will revisit the comparisons similar to those
considered in Chapter \ref{chapter2} to get more details on specific
differences among \emph{all} the pairs of groups -- what we call
\textbf{\emph{pair-wise comparisons}}. An issue is created when you
perform many tests simultaneously and we will augment our previous
methods with an adjusted method for pairwise comparisons to make our
results valid called \textbf{\emph{Tukey's Honest Significant
Difference}}.

To make this more concrete, we return to the original MockJury data,
making side-by-side boxplots and beanplots (Figure \ref{fig:Figure3-1}
as well summarizing the sentences for the three groups using
\texttt{favstats}.




\begin{Shaded}
\begin{Highlighting}[]
\KeywordTok{require}\NormalTok{(heplots)}
\KeywordTok{require}\NormalTok{(mosaic)}
\KeywordTok{data}\NormalTok{(MockJury)}
\KeywordTok{par}\NormalTok{(}\DataTypeTok{mfrow=}\KeywordTok{c}\NormalTok{(}\DecValTok{1}\NormalTok{,}\DecValTok{2}\NormalTok{))}
\KeywordTok{boxplot}\NormalTok{(Years}\OperatorTok{~}\NormalTok{Attr,}\DataTypeTok{data=}\NormalTok{MockJury)}
\KeywordTok{require}\NormalTok{(beanplot)}
\KeywordTok{beanplot}\NormalTok{(Years}\OperatorTok{~}\NormalTok{Attr,}\DataTypeTok{data=}\NormalTok{MockJury,}\DataTypeTok{log=}\StringTok{""}\NormalTok{,}\DataTypeTok{col=}\StringTok{"bisque"}\NormalTok{,}\DataTypeTok{method=}\StringTok{"jitter"}\NormalTok{)}
\end{Highlighting}
\end{Shaded}

\begin{figure}
\centering
\includegraphics{GreenwoodBanner_files/figure-latex/Figure3-1-1.pdf}
\caption{\label{fig:Figure3-1}Boxplot and beanplot of the sentences (years) for the three
treatment groups.}
\end{figure}

\begin{Shaded}
\begin{Highlighting}[]
\KeywordTok{favstats}\NormalTok{(Years}\OperatorTok{~}\NormalTok{Attr,}\DataTypeTok{data=}\NormalTok{MockJury)}
\end{Highlighting}
\end{Shaded}

\begin{verbatim}
##           Attr min Q1 median   Q3 max     mean       sd  n missing
## 1    Beautiful   1  2      3  6.5  15 4.333333 3.405362 39       0
## 2      Average   1  2      3  5.0  12 3.973684 2.823519 38       0
## 3 Unattractive   1  2      5 10.0  15 5.810811 4.364235 37       0
\end{verbatim}

There are slight differences in the sample sizes in the three groups
with 37 \emph{Unattractive}, 38 \emph{Average} and 39 \emph{Beautiful}
group responses, providing a data set has a total sample size of
\(N=114\). The \emph{Beautiful} and \emph{Average} groups do not appear
to be very different with means of 4.33 and 3.97 years. In Chapter
\ref{chapter2}, we found moderate evidence regarding the difference in
\emph{Average}and \emph{Unattractive}. It is less clear whether we might
find evidence of a difference between \emph{Beautiful} and
\emph{Unattractive} groups since we are comparing means of 5.81 and 4.33
years. All the distributions appear to be right skewed with relatively
similar shapes. The variability in \emph{Average} and
\emph{Unattractive} groups seems like it could be slightly different
leading to an overall concern of whether the variability is the same in
all the groups.

\section{Linear model for One-Way ANOVA (cell-means and
reference-coding)}\label{section3-2}

We introduced the statistical model \(y_{ij} = \mu_j+\epsilon_j\) in
Chapter \ref{chapter2} for the situation with \(j = 1 \text{ or } 2\) to
denote a situation where there were two groups and, for the model that
is consistent with the alternative hypothesis, the means differed. Now
we have three groups and the previous model can be extended to this new
situation by allowing \(j\) to be 1, 2, or 3. Now that we have more than
two groups, we need to admit that what we were doing in Chapter
\ref{chapter2} was actually fitting what is called a
\textbf{\emph{linear model}}. The linear model assumes that the
responses follow a normal distribution with the linear model defining
the mean, all observations have the same variance, and the parameters
for the mean in the model enter linearly. This last condition is hard to
explain at this level of material -- it is sufficient to know that there
models where the parameters enter the model nonlinearly and that they
are beyond the scope of this course. The result of this constraint is
that we will be able to use the same general modeling framework for the
methods introduced in Chapters \ref{chapter3}, \ref{chapter4},
\ref{chapter6}, \ref{chapter7}, and \ref{chapter8}.

As in Chapter \ref{chapter2}, we have a null hypothesis that defines a
situation (and model) where all the groups have the same mean.
Specifically, the \textbf{\emph{null hypothesis}} in the general
situation with \(J\) groups (\(J\ge 2\)) is to have all the
\textbf{true} group means equal,

\[H_0:\mu_1 = \ldots \mu_J.\]

This defines a model where all the groups have the same mean so it can
be defined in terms of a single mean, \(\mu\), for the \(i^{th}\)
observation from the \(j^{th}\) group as \(y_{ij} = \mu+\epsilon_{ij}\).
This is not the model that most researchers want to be the final
description of their study as it implies no difference in the groups.
There is more caution required to specify the alternative hypothesis
with more than two groups. The \textbf{\emph{alternative hypothesis}}
needs to be the logical negation of this null hypothesis of all groups
having equal means; to make the null hypothesis false, we only need one
group to differ but more than one group could differ from the others.
Essentially, there are many ways to ``violate'' the null hypothesis so
we choose some delicate wording for the alternative hypothesis when
there are more than 2 groups. Specifically, we state the alternative as

\[H_A: \text{ Not all } \mu_j \text{ are equal}\]

or, in words, \textbf{at least one of the true means differs among the J
groups}. You will be attracted to trying to say that all means are
different in the alternative but we do not put this strict a requirement
in place to reject the null hypothesis. The alternative model allows all
the true group means to differ but does require that they differ with

\[{\color{red}{\mu_j}}+\epsilon_{ij}.\]

This linear model states that the response for the \(i^{th}\)
observation in the \(j^{th}\) group, \(\mathbf{y_{ij}}\), is modeled
with a group \(j\) (\(j=1, \ldots, J\)) population mean, \(\mu_j\), and
a random error for each subject in each group \(\epsilon_{ij}\), that we
assume follows a normal distribution and that all the random errors have
the same variance, \(\sigma^2\). We can write the assumption about the
random errors, often called the \textbf{\emph{normality assumption}}, as
\(\epsilon_{ij} \sim N(0,\sigma^2)\). There is a second way to write out
this model that allows extension to more complex models discussed below,
so we need a name for this version of the model. The model written in
terms of the \({\color{red}{\mu_j}}\text{'s}\) is called the
\textcolor{red}{\textbf{cell means model}} and is the easier version of
this model to understand.

One of the reasons we learned about beanplots is that it helps us
visually consider all the aspects of this model. In the right panel of
Figure \ref{fig:Figure3-1}, we can see the wider, bold horizontal lines
that provide the estimated group means. The bigger the differences in
the sample means, the more likely we are to find evidence against the
null hypothesis. You can also see the null model on the plot that
assumes all the groups have the same as displayed in the dashed
horizontal line at 4.7 years (the R code below shows the overall mean of
\emph{Years} is 4.7). While the hypotheses focus on the means, the model
also contains assumptions about the distribution of the responses --
specifically that the distributions are normal and that all the groups
have the same variability. As discussed previously, it appears that the
distributions are right skewed and the variability might not be the same
for all the groups. The boxplot provides the information about the skew
and variability but since it doesn't display the means it is not
directly related to the linear model and hypotheses we are considering.

\begin{Shaded}
\begin{Highlighting}[]
\KeywordTok{mean}\NormalTok{(MockJury}\OperatorTok{$}\NormalTok{Years)}
\end{Highlighting}
\end{Shaded}

\begin{verbatim}
## [1] 4.692982
\end{verbatim}

There is a second way to write out the One-Way ANOVA model that provides
a framework for extensions to more complex models described in Chapter
\ref{chapter4} and beyond. The other \textbf{\emph{parameterization}}
(way of writing out or defining) of the model is called the
\textcolor{purple}{\textbf{reference-coded model}} since it writes out
the model in terms of a\\
\textbf{\emph{baseline group}} and deviations from that baseline or
reference level. The reference-coded model for the \(i^{th}\) subject in
the \(j^{th}\) group is
\(y_{ij} =\color{purple}{\boldsymbol{\alpha + \tau_j}}+\epsilon_{ij}\)
where \(\color{purple}{\boldsymbol{\alpha}}\) (alpha) is the true mean
for the baseline group (first alphabetically) and the
\(\color{purple}{\boldsymbol{\tau_j}}\) (tau \(j\)) are the deviations
from the baseline group for group \(j\). The deviation for the baseline
group, \(\color{purple}{\boldsymbol{\tau_1}}\), is always set to 0 so
there are really just deviations for groups 2 through \(J\). The
equivalence between the two models can be seen by considering the mean
for the first, second, and \(J^{th}\) groups in both models:

\begin{tabular}{l|l|l}
\hline
 & Cell means: & Reference-coded:\\
\hline
Group 1: & \$\{\textbackslash{}color\{red\}\{\textbackslash{}mu\_1\}\}\$ & \$\{\textbackslash{}color\{purple\}\{\textbackslash{}boldsymbol\{\textbackslash{}alpha\}\}\}\$\\
\hline
Group 2: & \$\{\textbackslash{}color\{red\}\{\textbackslash{}mu\_2\}\}\$ & \$\{\textbackslash{}color\{red\}\{\textbackslash{}boldsymbol\{\textbackslash{}tau\_2\}\}\}\$\\
\hline
\$\textbackslash{}ldots\$ & \$\textbackslash{}ldots\$ & \$\textbackslash{}ldots\$\\
\hline
Group \$J\$: & \$\{\textbackslash{}color\{red\}\{\textbackslash{}mu\_J\}\}\$ & \$\{\textbackslash{}color\{purple\}\{\textbackslash{}boldsymbol\{\textbackslash{}tau\_J\}\}\}\$\\
\hline
\end{tabular}

The hypotheses for the reference-coded model are similar to those in the
cell-means coding except that they are defined in terms of the
deviations, \({\color{purple}{\boldsymbol{\tau_j}}}\). The null
hypothesis is that there is no deviation from the baseline for any group
-- that all the \({\color{purple}{\boldsymbol{\tau_j\text{'s}}}}=0\),

\[\boldsymbol{H_0: \tau_2=\ldots=\tau_J=0}.\]

The alternative hypothesis is that at least one of the deviations is not
0,

\[\boldsymbol{H_A:} \textbf{ Not all } \boldsymbol{\tau_j} \textbf{ equal } \bf{0}.\]

In this chapter, you are welcome to use either version (unless we
instruct you otherwise) but we have to use the reference-coding in
subsequent chapters. The next task is to learn how to use R's linear
model \texttt{lm} function to get estimates of the parameters in each
model, but first a quick review of these new ideas:

\textcolor{red}{\textbf{Cell Means Version}}

\begin{itemize}
\item
  \(H_0: {\color{red}{\mu_1=\ldots\mu_J}}\) ~~~~~~~ ~~~
  \(H_A: {\color{red}{\text{ Not all } \mu_j \text{ equal}}}\)
\item
  Null hypothesis in words: No difference in the true means between the
  groups.
\item
  Null model \(y_{ij} = \mu_j+\epsilon_{ij}\)
\item
  Alternative hypothesis in words: At least one of the true means
  differs between the groups.
\item
  Alternative model: \(y_{ij} = \color{red}{\mu_j}+\epsilon_{ij}.\)
\end{itemize}

\textcolor{purple}{\textbf{Reference-coded Version}}

\begin{itemize}
\item
  \(H_0: \color{purple}{\boldsymbol{\tau_2 \ldots \tau_J = 0}}\)
  ~~~~~~~~
  \(H_A: \color{purple}{\text{ Not all } \tau_j \text{ equal}}\)
\item
  Null hypothesis in words: No deviation of the true mean for any groups
  from the baseline group.
\item
  Null model: \(y_{ij} =\boldsymbol{\alpha} + \tau_j+\epsilon_{ij}\)
\item
  Alternative hypothesis in words: At least one of the true deviations
  is different from 0 or that at least one group has a different true
  mean than the baseline group.
\item
  Alternative model:
  \(y_{ij} =\color{purple}{\boldsymbol{\alpha + \tau_j}}+\epsilon_{ij}\)
\end{itemize}

In order to estimate the models discussed above, the \texttt{lm}
function is used. If you look closely in the code for the rest of the
book, any model for a quantitative response will use this function,
suggesting a common thread in the most commonly used statistical models.
The \texttt{lm} function continues to use the same format as previous
functions, \texttt{lm(Y\textasciitilde{}X,\ data=datasetname)}. It ends
up that this code will give you the reference-coded version of the model
by default (R thinks it is that important!). We want to start with the
cell-means version of the model, so we have to override the standard
technique and add a ``\texttt{-1}'' to the formula interface to tell R
that we want to the cell-means coding. Generally, this looks like
\texttt{lm(Y\textasciitilde{}X-1\ ,\ data=datasetname).} Once we fit a
model in R, the \texttt{summary} function run on the model provides a
useful ``summary'' of the model coefficients and a suite of other
potentially interesting information. When fitting this version of the
One-Way ANOVA model, you will find a row of output for each group
relating the \(\mu_j\text{'s}\). The output contains columns for an
estimate (\texttt{Estimate}), standard error (\texttt{Std.Error}),
\(t\)-value (\texttt{t\ value}), and p-value
(\texttt{Pr(\textgreater{}\textbar{}t\textbar{})}). We'll learn to use
all the output in the following material, but for now just focus on the
estimates of the parameters that the function provides that we put in
bold.

\begin{Shaded}
\begin{Highlighting}[]
\NormalTok{lm1 <-}\StringTok{ }\KeywordTok{lm}\NormalTok{(Years }\OperatorTok{~}\StringTok{ }\NormalTok{Attr}\OperatorTok{-}\DecValTok{1}\NormalTok{, }\DataTypeTok{data=}\NormalTok{MockJury)}
\KeywordTok{summary}\NormalTok{(lm1)}
\end{Highlighting}
\end{Shaded}

\begin{verbatim}
## 
## Call:
## lm(formula = Years ~ Attr - 1, data = MockJury)
## 
## Residuals:
##     Min      1Q  Median      3Q     Max 
## -4.8108 -2.8108 -0.9737  2.1892 10.6667 
## 
## Coefficients:
##                  Estimate Std. Error t value Pr(>|t|)    
## AttrBeautiful      4.3333     0.5730   7.563 1.23e-11 ***
## AttrAverage        3.9737     0.5805   6.845 4.41e-10 ***
## AttrUnattractive   5.8108     0.5883   9.878  < 2e-16 ***
## ---
## Signif. codes:  0 '***' 0.001 '**' 0.01 '*' 0.05 '.' 0.1 ' ' 1
## 
## Residual standard error: 3.578 on 111 degrees of freedom
## Multiple R-squared:  0.6449, Adjusted R-squared:  0.6353 
## F-statistic: 67.21 on 3 and 111 DF,  p-value: < 2.2e-16
\end{verbatim}

\begin{longtable}[]{@{}ccccc@{}}
\toprule
\begin{minipage}[b]{0.27\columnwidth}\centering\strut
~\strut
\end{minipage} & \begin{minipage}[b]{0.13\columnwidth}\centering\strut
Estimate\strut
\end{minipage} & \begin{minipage}[b]{0.16\columnwidth}\centering\strut
Std. Error\strut
\end{minipage} & \begin{minipage}[b]{0.12\columnwidth}\centering\strut
t value\strut
\end{minipage} & \begin{minipage}[b]{0.12\columnwidth}\centering\strut
Pr(\textgreater{}\textbar{}t\textbar{})\strut
\end{minipage}\tabularnewline
\midrule
\endhead
\begin{minipage}[t]{0.27\columnwidth}\centering\strut
\textbf{AttrBeautiful}\strut
\end{minipage} & \begin{minipage}[t]{0.13\columnwidth}\centering\strut
\textbf{4.33}\strut
\end{minipage} & \begin{minipage}[t]{0.16\columnwidth}\centering\strut
0.573\strut
\end{minipage} & \begin{minipage}[t]{0.12\columnwidth}\centering\strut
7.56\strut
\end{minipage} & \begin{minipage}[t]{0.12\columnwidth}\centering\strut
1.23e-11\strut
\end{minipage}\tabularnewline
\begin{minipage}[t]{0.27\columnwidth}\centering\strut
\textbf{AttrAverage}\strut
\end{minipage} & \begin{minipage}[t]{0.13\columnwidth}\centering\strut
\textbf{3.97}\strut
\end{minipage} & \begin{minipage}[t]{0.16\columnwidth}\centering\strut
0.58\strut
\end{minipage} & \begin{minipage}[t]{0.12\columnwidth}\centering\strut
6.85\strut
\end{minipage} & \begin{minipage}[t]{0.12\columnwidth}\centering\strut
4.41e-10\strut
\end{minipage}\tabularnewline
\begin{minipage}[t]{0.27\columnwidth}\centering\strut
\textbf{AttrUnattractive}\strut
\end{minipage} & \begin{minipage}[t]{0.13\columnwidth}\centering\strut
\textbf{5.81}\strut
\end{minipage} & \begin{minipage}[t]{0.16\columnwidth}\centering\strut
0.588\strut
\end{minipage} & \begin{minipage}[t]{0.12\columnwidth}\centering\strut
9.88\strut
\end{minipage} & \begin{minipage}[t]{0.12\columnwidth}\centering\strut
6.86e-17\strut
\end{minipage}\tabularnewline
\bottomrule
\end{longtable}

In general, we denote estimated parameters with a hat over the parameter
of interest to show that it is an estimate. For the true mean of group
\(j\), \(\mu_j\), we estimate it with \(\hat{\mu}_j\), which is just the
sample mean for group \(j\), \(\bar{x}_j\). The model suggests an
estimate for each observation that we denote as \(\hat{y}_{ij}\) that we
will also call a \textbf{\emph{fitted value}} based on the model being
considered. The three estimates are bolded in the previous output, with
the same estimate used for all observations in the same group. R tries
to help you to sort out which row of output corresponds to which group
by appending the group name l with the variable name. Here, the variable
name was \texttt{Attr} and the first group alphabetically was
\emph{Beautiful}, so R provides a row labeled \texttt{AttrBeautiful}
with an estimate of 4.3333. The sample means from the three groups can
be seen to directly match that and the other two results.

\begin{Shaded}
\begin{Highlighting}[]
\KeywordTok{mean}\NormalTok{(Years }\OperatorTok{~}\StringTok{ }\NormalTok{Attr, }\DataTypeTok{data=}\NormalTok{MockJury)}
\end{Highlighting}
\end{Shaded}

\begin{verbatim}
##    Beautiful      Average Unattractive 
##     4.333333     3.973684     5.810811
\end{verbatim}

The reference-coded version of the same model is more complicated but
ends up giving the same results once we understand what it is doing. It
uses a different parameterization to accomplish this so has different
model output. Here is the model summary:

\begin{Shaded}
\begin{Highlighting}[]
\NormalTok{lm2 <-}\StringTok{ }\KeywordTok{lm}\NormalTok{(Years }\OperatorTok{~}\StringTok{ }\NormalTok{Attr, }\DataTypeTok{data=}\NormalTok{MockJury)}
\KeywordTok{summary}\NormalTok{(lm2)}
\end{Highlighting}
\end{Shaded}

\begin{verbatim}
## 
## Call:
## lm(formula = Years ~ Attr, data = MockJury)
## 
## Residuals:
##     Min      1Q  Median      3Q     Max 
## -4.8108 -2.8108 -0.9737  2.1892 10.6667 
## 
## Coefficients:
##                  Estimate Std. Error t value Pr(>|t|)    
## (Intercept)        4.3333     0.5730   7.563 1.23e-11 ***
## AttrAverage       -0.3596     0.8157  -0.441   0.6601    
## AttrUnattractive   1.4775     0.8212   1.799   0.0747 .  
## ---
## Signif. codes:  0 '***' 0.001 '**' 0.01 '*' 0.05 '.' 0.1 ' ' 1
## 
## Residual standard error: 3.578 on 111 degrees of freedom
## Multiple R-squared:  0.04754,    Adjusted R-squared:  0.03038 
## F-statistic:  2.77 on 2 and 111 DF,  p-value: 0.067
\end{verbatim}

The estimated model coefficients are \(\hat{\alpha} = 4.333\) years,
\(\hat{tau}_2 =-0.3596\) years, \(\hat{\tau}_3=1.4775\) years where R
selected group 1 for \emph{Beautiful}, 2 for \emph{Average}, and 3 for
\emph{Unattractive}. The way you can figure out the baseline group
(group 1 is \emph{Beautiful} here) is to see which category label is
\emph{not present} in the output. \textbf{The baseline level is
typically the first group label alphabetically}, but you should always
check this. Based on these definitions, there are interpretations
available for each coefficient. For \(\hat{\alpha} = 4.333\) years, this
is an estimate of the mean sentencing time for the
\emph{Beautiful}group. \(\hat{\tau}_2 =-0.3596\) years is the deviation
of the \emph{Average} group's mean from the \emph{Beautiful} groups mean
(specifically, it is \(0.36\) years lower). Finally,
\(\hat{\tau}_3=1.4775\) years tells us that the \emph{Unattractive}
group mean sentencing time is 1.48 years higher than time. These
interpretations lead directly to reconstructing the estimated means for
each group by combining the baseline and pertinent deviations as shown
in Table \ref{tab:Table3-1}.




\begin{table}

\caption{\label{tab:Table3-1}Constructing group mean estimates from the reference-coded
linear model estimates.}
\centering
\begin{tabular}[t]{l|l|l}
\hline
Group & Formula & Estimates\\
\hline
Beautiful & \$\textbackslash{}hat\{\textbackslash{}alpha\}\$ & **4.3333** years\\
\hline
Average & \$\textbackslash{}hat\{\textbackslash{}alpha\}+\textbackslash{}hat\{\textbackslash{}tau\}\_2\$ & 4.3333 - 0.3596 = **3.974** years\\
\hline
Unattractive & \$\textbackslash{}hat\{\textbackslash{}alpha\}+\textbackslash{}hat\{\textbackslash{}tau\}\_3\$ & 4.3333 + 1.4775 = **5.811** years\\
\hline
\end{tabular}
\end{table}

We can also visualize the results of our linear models using what are
called \textbf{\emph{term-plots}} or \textbf{\emph{effect-plots}} (from
the \texttt{effects} package; Fox, 2003) as displayed in Figure
\ref{fig:Figure3-2}. We don't want to use the word ``effect'' for these
model components unless we have random assignment in the study design so
we generically call these \textbf{\emph{term-plots}} as they display
terms or components from the model in hopefully useful ways to aid in
model interpretation even in the presence of complicated model
parameterizations. Specifically, these plots take an estimated model and
show you its estimates along with 95\% confidence intervals generated by
the linear model. To make this plot, you need to install and load the
\texttt{effects} package and then use \texttt{plot(allEffects(...))}
functions together on the \texttt{lm} object called \texttt{lm2} that
was estimated above. You can find the correspondence between the
displayed means and the estimates that were constructed in Table
\ref{tab:Table3-1}.




\begin{Shaded}
\begin{Highlighting}[]
\KeywordTok{require}\NormalTok{(effects)}
\KeywordTok{plot}\NormalTok{(}\KeywordTok{allEffects}\NormalTok{(lm2))}
\end{Highlighting}
\end{Shaded}

\begin{figure}
\centering
\includegraphics{GreenwoodBanner_files/figure-latex/Figure3-2-1.pdf}
\caption{\label{fig:Figure3-2}Plot of the estimated group mean sentences from the
reference-coded model for the MockJury data.}
\end{figure}

In order to assess evidence for having different means for the groups,
we will compare either of the previous models (cell-means or
reference-coded) to a null model based on the null hypothesis
(\(H_0: \mu_1 = \ldots = \mu_J\)) which implies a model of
\(\color{red}{y_{ij} = \mu_j}+\epsilon_{ij}\) in the cell-means version
where \({\color{red}{\mu}}\) is a common mean for all the observations.
We will call this the \textcolor{red}{\textbf{mean-only}} model since it
only has a single mean in it. In the reference-coding version of the
model, we have a null hypothesis that
\(H_0: \tau_2 = \ldots = \tau_J = 0\), so the ``mean-only'' model is
\(\color{purple}{y_{ij} =\boldsymbol{\alpha}+\epsilon_{ij}}\) with
\(\color{purple}{\boldsymbol{\alpha}}\) having the same definition as
\(\color{red}{\mu}\) for the cell means model -- it forces a common
value for the mean for all the groups. Moving from the
\emph{reference-coded} model to the \emph{mean-only} model is also an
example of a situation where we move from a ``full'' model to a
``reduced'' model by setting some coefficients in the ``full'' model to
0 and, by doing this, get a simpler or ``reduced'' model. Simple models
can be good as they are easier to groups that suggests no difference in
the groups is not a very exciting result in most, but not all,
situations\footnote{Suppose we were doing environmental monitoring and
  were studying asbestos levels in soils. We might be hoping that the
  mean-only model were reasonable to use if the groups being compared
  were in remediated areas and in areas known to have never been
  contaminated.}. In order for R to provide results for the mean-only
model, we remove the grouping variable, \texttt{Attr}, from the model
formula and just include a ``1''. The \texttt{(Intercept)} row of the
output provides the estimate for the mean-only model as a reduced model
from either the cell-means or reference-coded models when we assume that
the mean is the same for all groups:

\begin{Shaded}
\begin{Highlighting}[]
\NormalTok{lm3 <-}\StringTok{ }\KeywordTok{lm}\NormalTok{(Years }\OperatorTok{~}\StringTok{ }\DecValTok{1}\NormalTok{, }\DataTypeTok{data=}\NormalTok{MockJury)}
\KeywordTok{summary}\NormalTok{(lm3)}
\end{Highlighting}
\end{Shaded}

\begin{verbatim}
## $coefficients
##             Estimate Std. Error  t value     Pr(>|t|)
## (Intercept) 4.692982  0.3403532 13.78857 5.765681e-26
\end{verbatim}

This model provides an estimate of the common mean for all observations
of \(4.693 = \hat{\mu} = \hat{\alpha}\) years. This value also is the
dashed, horizontal line in the beanplot in Figure \ref{fig:Figure3-1}.
Some people call this mean-only estimate the grand or overall mean.

\section{One-Way ANOVA Sums of Squares, Mean Squares, and
F-test}\label{section3-3}

The previous discussion showed two ways of parameterizing models for the
One-Way ANOVA model and getting estimates from output but still hasn't
addressed how to assess evidence related to whether the observed
differences in the means among the groups is ``real''. In this section,
we develop what is called the \textbf{\emph{ANOVA F-test}} that provides
a method of aggregating the differences among the means of 2 or more
groups and testing our null hypothesis of no difference in the means vs
the alternative. In order to develop the test, some additional notation
is needed. The sample size in each group is denoted \(n_j\) and the
total sample size is
\(\boldsymbol{N=\Sigma n_j = n_1 + n_2 + \ldots + n_J}\) where
\(\Sigma\) (capital sigma) means ``add up over whatever follows''. An
estimated \textbf{\emph{residual}} (\(e_{ij}\)) is the difference
between an observation, \(y_{ij}\), and the model estimate,
\(\hat{y}_{ij} = \hat{\mu}_j\), for that observation,
\(y_{ij}-\hat{y}_{ij} = e_{ij}\). It is basically what is left over that
the mean part of the model (\(\hat{\mu}_{j}\)) does not explain. It is
also a window into how ``good'' the model might be.

Consider the four different fake results for a situation with four
groups (\(J=4\)) displayed in Figure \ref{fig:Figure3-3}. Which of the
different results shows the most and least evidence of differences in
the means? In trying to answer this, think about both how different the
means are (obviously important) and how variable the results are around
the mean. These situations were created to have the same means in
Scenarios 1 and 2 as well as matching means in Scenarios 3 and 4. The
variability around the means matches by shading (lighter or darker). In
Scenarios 1 and 2, the differences in the means is smaller than in the
other two results. But Scenario 2 should provide more evidence of what
little difference in present than Scenario 1 because it has less
variability around the means. The best situation for finding group
differences here is Scenario 4 since it has the largest difference in
the means and the least variability around those means. Our test
statistic somehow needs to allow a comparison of the variability in the
means to the overall variability to help us get results that reflect
that Scenario 4 has the strongest evidence of a difference and Scenario
1 would have the least.





\begin{figure}
\centering
\includegraphics{GreenwoodBanner_files/figure-latex/Figure3-3-1.pdf}
\caption{\label{fig:Figure3-3}Demonstration of different amounts of difference in means
relative to variability. Scenarios have same means in rows and same
variance around means in columns of plot.}
\end{figure}

The statistic that allows the comparison of relative amounts of
variation is called the \textbf{\emph{ANOVA F-statistic}} . It is
developed using \textbf{\emph{sums of squares}} which are measures of
total variation like are used in the numerator of the standard deviation
(\(\Sigma_1^N(y_i-\bar{y})^2\)) that took all the observations,
subtracted the mean, squared the differences, and then added up the
results over all the observations to generate a measure of total
variability. With multiple groups, we will focus on decomposing that
total variability (\textbf{\emph{Total Sums of Squares}}) into
variability among the means (we'll call this \textbf{\emph{Explanatory
Variable}} \(\mathbf{A}\textbf{'s}\) \textbf{\emph{Sums of Squares}})
and variability in the residuals or errors ( \textbf{\emph{Error Sums of
Squares}}). We define each of these quantities in the One-Way ANOVA
situation as follows:

\begin{itemize}
\item
  \(\textbf{SS}_{\textbf{Total}} =\) Total Sums of Squares
  \(= \Sigma^J_{j=1}\Sigma^{n_j}_{i=1}(y_{ij}-\bar{\bar{y}})^2\)

  \begin{itemize}
  \item
    This is the total variation in the responses around the overall or
    \textbf{\emph{grand mean}} (\(\bar{\bar{y}}\), the estimated mean
    for all the observations and available from the mean-only model).
  \item
    By summing over all \(n_j\) observations in each group,
    \(\Sigma^{n_j}_{i=1}(\ )\), and then adding those results up across
    the groups, \(\Sigma^J_{j=1}(\ )\), we accumulate the variation
    across all \(N\) observations.
  \item
    Note: this is the residual variation if the null model is used, so
    there is no further decomposition possible for that model.
  \item
    This is also equivalent to the numerator of the sample variance,
    \(\Sigma^{N}_{1}(y_{i}-\bar{y})^2\) which is what you get when you
    ignore the information on the potential differences in the groups.
  \end{itemize}
\item
  \(\textbf{SS}_{\textbf{A}} =\) Explanatory Variable \emph{A}'s Sums of
  Squares
  \(=\Sigma^J_{j=1}\Sigma^{n_j}_{i=1}(\bar{y}_{i}-\bar{\bar{y}})^2 =\Sigma^J_{j=1}n_j(\bar{y}_{i}-\bar{\bar{y}})^2\)

  \begin{itemize}
  \item
    This is the variation in the group means around the grand mean based
    on the explanatory variable \(A\).
  \item
    Also called sums of squares for the treatment, regression, or model.
  \end{itemize}
\item
  \(\textbf{SS}_E =\) Error (Residual) Sums of Squares
  \(=\Sigma^J_{j=1}\Sigma^{n_j}_{i=1}(y_{ij}-\bar{y})^2 =\Sigma^J_{j=1}\Sigma^{n_j}_{i=1}(e_{ij})^2\)

  \begin{itemize}
  \item
    This is the variation in the responses around the group means.
  \item
    Also called the sums of squares for the residuals, with the second
    version of the formula showing that it is just the squared residuals
    added up across all the observations.
  \end{itemize}
\end{itemize}

The possibly surprising result given the mass of notation just presented
is that the total sums of squares is \textbf{ALWAYS} equal to the sum of
explanatory variable \(A\text{'s}\) sum of squares and the error sums of
squares,

\[\textbf{SS}_{\textbf{Total}} \mathbf{=} \textbf{SS}_\textbf{A} \mathbf{+} \textbf{SS}_\textbf{E}.\]

This equality means that if the \(\textbf{SS}_\textbf{A}\) goes up, then
the \(\textbf{SS}_\textbf{E}\) must go down if
\(\textbf{SS}_{\textbf{Total}}\) remains the same. This result is called
the \textbf{\emph{sums of squares decomposition formula}}. We use these
results to build our test statistic and organize this information in
what is called an \textbf{\emph{ANOVA table}}. The ANOVA table is
generated using the \texttt{anova} function applied to the
reference-coded model, \texttt{lm2} :

\begin{Shaded}
\begin{Highlighting}[]
\NormalTok{lm2<-}\KeywordTok{lm}\NormalTok{(Years }\OperatorTok{~}\StringTok{ }\NormalTok{Attr, }\DataTypeTok{data=}\NormalTok{MockJury)}
\KeywordTok{anova}\NormalTok{(lm2)}
\end{Highlighting}
\end{Shaded}

\begin{verbatim}
## Analysis of Variance Table
## 
## Response: Years
##            Df  Sum Sq Mean Sq F value Pr(>F)  
## Attr        2   70.94  35.469    2.77  0.067 .
## Residuals 111 1421.32  12.805                 
## ---
## Signif. codes:  0 '***' 0.001 '**' 0.01 '*' 0.05 '.' 0.1 ' ' 1
\end{verbatim}

Note that the ANOVA table has a row labelled \texttt{Attr}, which
contains information for the grouping variable (we'll generally refer to
this as explanatory variable \(A\) but here it is the picture group that
was randomly assigned), and a row labelled \texttt{Residuals}, which is
synonymous with ``Error''. The Sums of Squares (SS) are available in the
\texttt{Sum\ Sq} column. It doesn't show a row for ``Total'' but the
\(\textbf{SS}_{\textbf{Total}} \mathbf{=} \textbf{SS}_\textbf{A} \mathbf{+} \textbf{SS}_\textbf{E} = 1492.26\).

\begin{Shaded}
\begin{Highlighting}[]
\FloatTok{70.94} \OperatorTok{+}\StringTok{ }\FloatTok{1421.32}
\end{Highlighting}
\end{Shaded}

\begin{verbatim}
## [1] 1492.26
\end{verbatim}

It may be easiest to understand the \emph{sums of squares decomposition}
by connecting it to our permutation ideas. In a permutation situation,
the total variation (\(SS_{Total}\)) cannot change -- it is the same
responses varying around the grand mean. However, the amount of
variation attributed to variation among the means and in the residuals
can change if we change which observations go with which group. In
Figure \ref{fig:Figure3-4} (panel a), the means, sums of squares, and
95\% confidence intervals for each mean are displayed for the three
treatment levels from the original prisoner rating data. Three permuted
versions of the data set are summarized in panels (b), (c), and (d). The
\(\text{SS}_A\) is 70.9 in the real data set and between 6.6 and 11 in
the permuted data sets. If you had to pick among the plots for the one
with the most evidence of a difference in the means, you hopefully would
pick panel (a). This visual ``unusualness'' suggests that this observed
result is unusual relative to the possibilities under permutations,
which are, again, the possibilities tied to having the null hypothesis
being true. But note that the differences here are not that great
between these three permuted data sets and the real one. It is likely
that at least some of you might have selected panel (d) as also looking
like it shows some evidence of differences (maybe not the most?) as it
also looks like it shows some evidence differences.

One way to think about \(\textbf{SS}_\textbf{A}\) is that it is a
function that converts the variation in the group means into a single
value. This makes it a reasonable test statistic in a permutation
testing context. By comparing the observed \(\text{SS}_A =\) 70.9 to the
permutation results of 6.5, 9.7, and 40.5 we see that the observed
result is much more extreme than the three alternate versions. In
contrast to our previous test statistics where positive and negative
differences were possible, \(\text{SS}_A\) is always positive with a
value of 0 corresponding to no variation in the means. The larger the
\(\text{SS}_A\), the more variation there is in the means. The
permutation p-value for the alternative hypothesis of \textbf{some} (not
of greater or less than!) difference in the true means of the groups
will involve counting the number of permuted \(SS_A^*\) results that are
larger than what we observed.








\begin{verbatim}
## [1] 70.93836
\end{verbatim}

\begin{figure}
\centering
\includegraphics{GreenwoodBanner_files/figure-latex/Figure3-4-1.pdf}
\caption{\label{fig:Figure3-4}Plot of means and 95\% confidence intervals for the three
groups for the real data (a) and three different permutations of the
treatment labels to the same responses in (b), (c), and (d). Note that
SSTotal is always the same but the different amounts of variation
associated with the means (SSA) or the errors (SSE) changes in
permutation.}
\end{figure}

To do a permutation test, we need to be able to calculate and extract
the \(\text{SS}_A\) value. In the ANOVA table, it is in the first row
and is the second number and we can use the bracket, \texttt{{[},\ {]}},
referencing to extract that number from the ANOVA table that
\texttt{anova} produces with
\texttt{anova(lm(Years\textasciitilde{}Attr,\ data=MockJury)){[}1,\ 2{]}}.
We'll store the observed value of \(\text{SS}_A\) in \texttt{Tobs},
reusing some ideas from Chapter @ref\{chapter2\}.

\begin{Shaded}
\begin{Highlighting}[]
\NormalTok{Tobs <-}\StringTok{ }\KeywordTok{anova}\NormalTok{(}\KeywordTok{lm}\NormalTok{(Years}\OperatorTok{~}\NormalTok{Attr,}\DataTypeTok{data=}\NormalTok{MockJury))[}\DecValTok{1}\NormalTok{,}\DecValTok{2}\NormalTok{]; Tobs}
\end{Highlighting}
\end{Shaded}

\begin{verbatim}
## [1] 70.93836
\end{verbatim}

The following code performs the permutations \texttt{B=1,000} times
using the \texttt{shuffle} function, builds up a vector of results in
\texttt{Tobs}, and then makes a plot of the resulting permutation
distribution:







\begin{Shaded}
\begin{Highlighting}[]
\KeywordTok{par}\NormalTok{(}\DataTypeTok{mfrow=}\KeywordTok{c}\NormalTok{(}\DecValTok{1}\NormalTok{,}\DecValTok{2}\NormalTok{))}
\NormalTok{B<-}\StringTok{ }\DecValTok{1000}
\NormalTok{Tstar<-}\KeywordTok{matrix}\NormalTok{(}\OtherTok{NA}\NormalTok{,}\DataTypeTok{nrow=}\NormalTok{B)}
\ControlFlowTok{for}\NormalTok{ (b }\ControlFlowTok{in}\NormalTok{ (}\DecValTok{1}\OperatorTok{:}\NormalTok{B))\{}
\NormalTok{  Tstar[b]<-}\KeywordTok{anova}\NormalTok{(}\KeywordTok{lm}\NormalTok{(Years}\OperatorTok{~}\KeywordTok{shuffle}\NormalTok{(Attr),}\DataTypeTok{data=}\NormalTok{MockJury))[}\DecValTok{1}\NormalTok{,}\DecValTok{2}\NormalTok{]}
\NormalTok{  \}}
\KeywordTok{hist}\NormalTok{(Tstar,}\DataTypeTok{labels=}\NormalTok{T,}\DataTypeTok{ylim=}\KeywordTok{c}\NormalTok{(}\DecValTok{0}\NormalTok{,}\DecValTok{550}\NormalTok{))}
\KeywordTok{abline}\NormalTok{(}\DataTypeTok{v=}\NormalTok{Tobs,}\DataTypeTok{col=}\StringTok{"red"}\NormalTok{,}\DataTypeTok{lwd=}\DecValTok{3}\NormalTok{)}
\KeywordTok{plot}\NormalTok{(}\KeywordTok{density}\NormalTok{(Tstar),}\DataTypeTok{main=}\StringTok{"Density curve of Tstar"}\NormalTok{)}
\KeywordTok{abline}\NormalTok{(}\DataTypeTok{v=}\NormalTok{Tobs,}\DataTypeTok{col=}\StringTok{"red"}\NormalTok{,}\DataTypeTok{lwd=}\DecValTok{3}\NormalTok{)}
\end{Highlighting}
\end{Shaded}

\begin{figure}
\centering
\includegraphics{GreenwoodBanner_files/figure-latex/Figure3-5-1.pdf}
\caption{\label{fig:Figure3-5}Histogram and density curve of permutation distribution of
\(\text{SS}_A\) with the observed value of \(\text{SS}_A\) displayed as
a bold, vertical line. The proportion of results that are larger than
the observed value of \(\text{SS}_A\) provides an estimate of the
p-value.}
\end{figure}

The right-skewed distribution (Figure \ref{fig:Figure3-5}) contains the
distribution of \(\text{SS}_A\text{'s}\) under permutations (where all
the groups are assumed to be equivalent under the null hypothesis).
While the observed result is larger than many of the
\(\text{SS}_A\text{'s}\), there are also many permuted results that are
much larger than observed. The proportion of permuted results that
exceed the observed value is found using \texttt{pdata} as before,
except only for the area to the right of the observed result. We know
that \texttt{Tobs} will always be positive so no absolute values are
required here.

\begin{Shaded}
\begin{Highlighting}[]
\KeywordTok{pdata}\NormalTok{(Tstar,Tobs,}\DataTypeTok{lower.tail=}\NormalTok{F)}
\end{Highlighting}
\end{Shaded}

\begin{verbatim}
## [1] 0.072
\end{verbatim}

This provides a permutation-based p-value of 0.072 and suggests marginal
evidence against the null hypothesis of no difference in the true means.
We would interpret this p-value as saying that there is a 7.2\% chance
of getting a \(\text{SS}_A\) as large or larger than we observed, given
that the null hypothesis is true.

It ends up that some nice parametric statistical results are available
(if our assumptions are met) for the ratio of estimated variances, which
are called \textbf{\emph{Mean Squares}}. To turn sums of squares into
mean square (variance) estimates, we divide the sums of squares by the
amount of free information available. For example, remember the typical
variance estimator introductory statistics,
\(\Sigma^N_1(y_i-\bar{y})^2/(N-1)\)? Your instructor spent some time
trying various approaches to explaining why we have a denominator of
\(N-1\). The most useful for our purposes moving forward is that we
``lose'' one piece of information to estimate the mean and there are
\(N\) deviations around the single mean so we divide by \(N-1\). The
main point is that the sums of squares were divided by something and we
got an estimator for the variance, here of the observations.

Now consider
\(\text{SS}_E = \Sigma^J_{j=1}\Sigma^{n_j}_{i=1}(y_i-\bar{y})^2\) which
still has \(N\) deviations but it varies around the \(J\) means, so the

\[\textbf{Mean Square Error} = \text{MS}_E = \text{SS}_E/(N-J).\]

Basically, we lose \(J\) pieces of information in this calculation
because we have to estimate \(J\) means.

The similar calculation of the \textbf{\emph{Mean Square for variable}}
\(\mathbf{A}\) (\(\text{MS}_A\)) is harder to see in the formula
(\(\text{SS}_A = \Sigma^J_{j=1}n_j(\bar{y}_i-\bar{\bar{y}})^2\)), but
the same reasoning can be used to understand the denominator for forming
\(\text{MS}_A\): there are \(J\) means that vary around the grand mean
so

\[\text{MS}_A = \text{SS}_A/(J-1).\]

In summary, the two mean squares are simply:

\begin{itemize}
\item
  \(\text{MS}_A = \text{SS}_A/(J-1)\), which estimates the variance of
  the group means around the grand mean.
\item
  \(\text{MS}_{\text{Error}} = \text{SS}_{\text{Error}}/(N-J)\), which
  estimates the variation of the errors around the group means.
\end{itemize}

These results are put together using a ratio to define the
\textbf{\emph{ANOVA F-statistic}} (also called the
\textbf{\emph{F-ratio}} ) as

\[F=\text{MS}_A/\text{MS}_{\text{Error}}.\]

If the variability in the means is ``similar'' to the variability in the
residuals, the statistic would have a value around 1. If that
variability is similar then there be no evidence of a difference in the
means. If the \(\text{MS}_A\) is much larger than the \(\text{MS}_E\),
the \(F\)-statistic will provide evidence against the null hypothesis.
The ``size'' of the \(F\)-statistic is formalized by finding the
p-value. The \(F\)-statistic, if assumptions discussed below are met and
we assume the null hypothesis is true, follows what is called an
\(F\)-distribution. The \textbf{\emph{F-distribution}} is a right-skewed
distribution whose shape is defined by what are called the
\textbf{\emph{numerator degrees of freedom}} (\(J-1\)) and the
\textbf{\emph{denominator degrees of freedom}} (\(N-J\)). These names
correspond to the values that we used to calculate the mean squares and
where in the \(F\)-ratio each mean square was used; \(F\)-distributions
are denoted by their degrees of freedom using the convention of \(F\)
(\emph{numerator df}, \emph{denominator df}). Some examples of different
\(F\)-distributions are displayed for you in Figure \ref{fig:Figure3-6}.

The characteristics of the F-distribution can be summarized as:

\begin{itemize}
\item
  Right skewed,
\item
  Nonzero probabilities for values greater than 0,
\item
  Its shape changes depending on the \textbf{numerator} and
  \textbf{denominator DF}, and
\item
  \textbf{Always use the right-tailed area for p-values.}
\end{itemize}






\begin{figure}
\centering
\includegraphics{GreenwoodBanner_files/figure-latex/Figure3-6-1.pdf}
\caption{\label{fig:Figure3-6}Density curves of four different \(F\)-distributions. Upper
left is an \(F(2, 111)\), upper right is \(F(2, 10)\), lower left is
\(F(6, 10)\), and lower right is \(F(6, 111)\). P-values are found using
the areas to the right of the observed \(F\)-statistic value.}
\end{figure}

Now we are ready to discuss an ANOVA table since we know about each of
its components. Note the general format of the ANOVA table is\footnote{Make
  sure you can work from left to right and up and down to fill in the
  ANOVA table given just the necessary information to determine the
  other components -- there is always a question like this on the
  exam\ldots{}}:



\begin{table}

\caption{\label{tab:Table3-2}General One-Way ANOVA table.}
\centering
\begin{tabular}[t]{l|l|l|l|l|l}
\hline
Source & DF & Sums of Squares & Mean Squares & F-ratio & P-value\\
\hline
Variable A & \$J-1\$ & \$\textbackslash{}text\{SS\}\_A\$ & \$\textbackslash{}text\{MS\}\_A=\textbackslash{}text\{SS\}\_A/(J-1)\$ & \$F=\textbackslash{}text\{MS\}\_A/\textbackslash{}text\{MS\}\_E\$ & Right tail of \$F(J-1,N-J)\$\\
\hline
Residuals & \$N-J\$ & \$\textbackslash{}text\{SS\}\_E\$ & \$\textbackslash{}text\{MS\}\_E = \textbackslash{}text\{SS\}\_E/(N-J)\$ &  & \\
\hline
Total & \$N-1\$ & \$\textbackslash{}text\{SS\}\_\{\textbackslash{}text\{Total\}\}\$ &  &  & \\
\hline
\end{tabular}
\end{table}

The table is oriented to help you reconstruct the \(F\)-ratio from each
of its components. The output from R is similar although it does not
provide the last row and sometimes switches the order of columns. The R
version of the table for the type of picture effect (\texttt{Attr}) with
\(J=3\) levels and \(N=114\) observations, repeated from above, is:

\begin{Shaded}
\begin{Highlighting}[]
\KeywordTok{anova}\NormalTok{(lm2)}
\end{Highlighting}
\end{Shaded}

\begin{verbatim}
## Analysis of Variance Table
## 
## Response: Years
##            Df  Sum Sq Mean Sq F value Pr(>F)  
## Attr        2   70.94  35.469    2.77  0.067 .
## Residuals 111 1421.32  12.805                 
## ---
## Signif. codes:  0 '***' 0.001 '**' 0.01 '*' 0.05 '.' 0.1 ' ' 1
\end{verbatim}

The p-value from the \(F\)-distribution is 0.067. We can verify this
result using the observed \(F\)-statistic of 2.77 (which came from
taking the ratio of the two mean squares, F=35.47/12.8) which follows an
\(F(2, 111)\) distribution if the null hypothesis is true and some other
assumptions are met. Using the \texttt{pf} function provides us with
areas in the specified \(F\)-distribution with the \texttt{df1} provided
to the function as the numerator \emph{df} and \texttt{df2} as the
denominator \emph{df} and \texttt{lower.tail=F} reflecting our desire
for a right tailed area.

\begin{Shaded}
\begin{Highlighting}[]
\KeywordTok{pf}\NormalTok{(}\FloatTok{2.77}\NormalTok{,}\DataTypeTok{df1=}\DecValTok{2}\NormalTok{,}\DataTypeTok{df2=}\DecValTok{111}\NormalTok{,}\DataTypeTok{lower.tail=}\NormalTok{F)}
\end{Highlighting}
\end{Shaded}

\begin{verbatim}
## [1] 0.06699803
\end{verbatim}

The result from the \(F\)-distribution using this parametric procedure
is similar to the p-value obtained using permutations with the test
statistic of the \(\text{SS}_A\), which was 70.9. The \(F\)-statistic
obviously is another potential test statistic to use as a test statistic
in a permutation approach, now that we know about it. We should check
that we get similar results from it with permutations as we did from
using \(\text{SS}_A\) as a permutation test test statistic. The
following code generates the permutation distribution for the
\(F\)-statistic (Figure \ref{fig:Figure3-7}) and assesses how unusual
the observed \(F\)-statistic of 2.77 was in this permutation
distribution. The only change in the code involves moving from
extracting \(\text{SS}_A\) to extracting the \(F\)-ratio which is in the
4th column of the \texttt{anova} output:





\begin{Shaded}
\begin{Highlighting}[]
\NormalTok{Tobs <-}\StringTok{ }\KeywordTok{anova}\NormalTok{(}\KeywordTok{lm}\NormalTok{(Years}\OperatorTok{~}\NormalTok{Attr,}\DataTypeTok{data=}\NormalTok{MockJury))[}\DecValTok{1}\NormalTok{,}\DecValTok{4}\NormalTok{]; Tobs}
\end{Highlighting}
\end{Shaded}

\begin{verbatim}
## [1] 2.770024
\end{verbatim}

\begin{Shaded}
\begin{Highlighting}[]
\KeywordTok{par}\NormalTok{(}\DataTypeTok{mfrow=}\KeywordTok{c}\NormalTok{(}\DecValTok{1}\NormalTok{,}\DecValTok{2}\NormalTok{))}
\NormalTok{B<-}\StringTok{ }\DecValTok{1000}
\NormalTok{Tstar<-}\KeywordTok{matrix}\NormalTok{(}\OtherTok{NA}\NormalTok{,}\DataTypeTok{nrow=}\NormalTok{B)}
\ControlFlowTok{for}\NormalTok{ (b }\ControlFlowTok{in}\NormalTok{ (}\DecValTok{1}\OperatorTok{:}\NormalTok{B))\{}
\NormalTok{  Tstar[b]<-}\KeywordTok{anova}\NormalTok{(}\KeywordTok{lm}\NormalTok{(Years}\OperatorTok{~}\KeywordTok{shuffle}\NormalTok{(Attr),}\DataTypeTok{data=}\NormalTok{MockJury))[}\DecValTok{1}\NormalTok{,}\DecValTok{4}\NormalTok{]}
\NormalTok{\}}

\KeywordTok{pdata}\NormalTok{(Tstar,Tobs,}\DataTypeTok{lower.tail=}\NormalTok{F)}
\end{Highlighting}
\end{Shaded}

\begin{verbatim}
## [1] 0.064
\end{verbatim}

\begin{Shaded}
\begin{Highlighting}[]
\KeywordTok{hist}\NormalTok{(Tstar,}\DataTypeTok{labels=}\NormalTok{T)}
\KeywordTok{abline}\NormalTok{(}\DataTypeTok{v=}\NormalTok{Tobs,}\DataTypeTok{col=}\StringTok{"red"}\NormalTok{,}\DataTypeTok{lwd=}\DecValTok{3}\NormalTok{)}
\KeywordTok{plot}\NormalTok{(}\KeywordTok{density}\NormalTok{(Tstar),}\DataTypeTok{main=}\StringTok{"Density curve of Tstar"}\NormalTok{)}
\KeywordTok{abline}\NormalTok{(}\DataTypeTok{v=}\NormalTok{Tobs,}\DataTypeTok{col=}\StringTok{"red"}\NormalTok{,}\DataTypeTok{lwd=}\DecValTok{3}\NormalTok{)}
\end{Highlighting}
\end{Shaded}

\begin{figure}
\centering
\includegraphics{GreenwoodBanner_files/figure-latex/Figure3-7-1.pdf}
\caption{\label{fig:Figure3-7}Histogram and density curve of the permutation distribution
of the F-statistic with bold, vertical line for observed value of the
test statistic of 2.77.}
\end{figure}

The permutation-based p-value is 0.064 which, again, matches the other
results closely. The first conclusion is that using a test statistic of
either the \(F\)-statistic or the \(\text{SS}_A\) provide similar
permutation results. However, we tend to favor using the \(F\)-statistic
because it is more commonly used in reporting ANOVA results, not because
it is any better in a permutation context.

It is also interesting to compare the permutation distribution for the
\(F\)-statistic and the parametric \(F(2, 111)\) distribution (Figure
\ref{fig:Figure3-8}). They do not match perfectly but are quite similar.
Some the differences around 0 are due to the behavior of the method used
to create the density curve and are not really a problem for the
methods. The similarity in the two curves explains why both methods give
similar results. In some situations, the correspondence will not be
quite so close.




\begin{figure}
\centering
\includegraphics{GreenwoodBanner_files/figure-latex/Figure3-8-1.pdf}
\caption{\label{fig:Figure3-8}Comparison of \(F(2, 111)\) (dashed line) and permutation
distribution (solid line).}
\end{figure}

So how can we rectify this result (\(\text{p-value}\approx 0.06\)) and
the Chapter \ref{chapter2} result that detected a difference between
\emph{Average} and \emph{Unattractive} with a
\(\text{p-value}\approx 0.03\)? I selected the two groups to compare in
Chapter \ref{chapter2} because they were furthest apart.
``Cherry-picking'' the comparison that is likely to be most different
creates a false sense of the real situation and inflates the Type I
error rate because of the selection. If the entire suite of pairwise
comparisons are considered, this result may lose some of its luster. In
other words, if we consider the suite of three pair-wise differences
(and the tests) implicit in comparing all of them, we may need stronger
evidence in the most different pair than a p-value of 0.033 to suggest
overall differences. In this situation, the \emph{Beautiful} and
\emph{Average} groups are not that different from each other so their
difference does not contribute much to the will revisit this topic and
consider a method that is statistically valid for performing all
possible pair-wise comparisons that is also consistent with our overall
test results.

\section{ANOVA model diagnostics including QQ-plots}\label{section3-4}

The requirements for a One-Way ANOVA \(F\)-test are similar to those
discussed in Chapter \ref{chapter2}, except that there are now \(J\)
groups instead of only 2. Specifically, the linear model assumes:

\begin{enumerate}
\def\labelenumi{\arabic{enumi}.}
\item
  \textbf{Independent observations},
\item
  \textbf{Equal variances}, and
\item
  \textbf{Normal distributions}.
\end{enumerate}

For assessing equal variances across the groups, it is best to use plots
to assess this. We can use boxplots and beanplots to compare the spreads
of the groups, which were provided in Figure \ref{fig:Figure3-1}. The
range and IQRs should be relatively similar across the groups if you do
not find evidence of a problem with this assumption. You should start
with noting how clear or big the violation of the assumption might be
but remember that there will always be some differences in the variation
among groups even if the true variability is exactly equal in the
populations. In addition to our direct plotting, there are some
diagnostic plots available from the \texttt{lm} function that can help
us more clearly assess potential violations of the previous assumptions.

We can obtain a suite of four diagnostic plots by using the
\texttt{plot} function on any linear model object that we have fit. To
get all the plots together in four panels we need to add the
\texttt{par(mfrow=c(2,\ 2))} command to tell R to make a graph with 4
panels\footnote{We have been using this function quite a bit to make
  multi-panel graphs but did not show you that line of code. But you
  need to use this command for linear model diagnostics or you won't get
  the plots we want from the model. And you really just need
  \texttt{plot(lm2)} but the \texttt{pch=16} option makes it easier to
  see some of the points in the plots.}.

\begin{Shaded}
\begin{Highlighting}[]
\KeywordTok{par}\NormalTok{(}\DataTypeTok{mfrow=}\KeywordTok{c}\NormalTok{(}\DecValTok{2}\NormalTok{,}\DecValTok{2}\NormalTok{))}
\KeywordTok{plot}\NormalTok{(lm2,}\DataTypeTok{pch=}\DecValTok{16}\NormalTok{)}
\end{Highlighting}
\end{Shaded}

There are two plots in Figure \ref{fig:Figure3-9} with useful
information for the equal variance assumption. The ``Residuals vs
Fitted'' panel in the top left displays the residuals
\((e_{ij} = y_{ij}-\hat{y}_ij)\) on the y-axis and the fitted values
\((\hat{y}_{ij})\) on the x-axis. This allows you to see if the
variability of the observations differs across the groups as a function
of the mean of the groups because all the observations in the same group
get the same fitted value, the mean of the group. In this plot, the
points seem to have fairly similar spreads at the fitted values for the
three groups with fitted values of 4, 4.3, and 6. The ``Scale-Location''
plot in the lower left panel has the same x-axis but the y-axis contains
the square-root of the absolute value of the standardized residuals. The
absolute value transforms all the residuals into a magnitude scale
(removing direction) and the square-root helps you see differences in
variability more accurately. The standardization scales them to have a
variance of 1 so help you in other displays to get a sense of how many
standard deviations you are away from the mean in the residual
distribution. The visual assessment is similar in the two plots -- you
want to consider whether it appears that the groups have somewhat
similar or noticeably different amounts of variability. If you see a
clear funnel shape in the Residuals vs Fitted or an increase or decrease
in the upper edge of points in the Scale-Location plot that may indicate
a violation of the constant variance assumption. Remember that some
variation across the groups is expected and is OK, but large differences
in spreads are problematic for all the procedures that involve linear
models. When discussing these results, you want to discuss how clearly
the differences in variation are and whether that \emph{shows a clear
violation of the assumption} of equal variance for all observations.
Like in hypothesis testing, you can't prove that you've met assumptions
based on a plot ``looking OK'', but you can say that there is no clear
evidence that the assumption is violated!



\begin{figure}
\centering
\includegraphics{GreenwoodBanner_files/figure-latex/Figure3-9-1.pdf}
\caption{\label{fig:Figure3-9}Default diagnostic plots for the linear model.}
\end{figure}

The linear model assumes that all the random errors (\(\epsilon_{ij}\))
follow a normal distribution. To gain insight into the validity of this
assumption, we can explore the original observations as displayed in the
beanplots, mentally subtracting off the differences in the means and
focusing on the shapes of the distributions of observations in each
group. These plots are especially good for assessing whether there is
there a skew or outliers present in each group. If so, by definition,
the normality assumption is violated. But our assumption is about the
distribution of all the errors after the remove the differences in the
means and so we want an overall assessment technique to understand how
reasonable our assumption is overall for our model. The residuals from
the entire model provide us with estimates of the random errors and if
the normality assumption is met, then the residuals all-together should
approximately follow a normal distribution. The \textbf{\emph{Normal Q-Q
Plot}} in upper right panel of Figure \ref{fig:Figure3-9} is a direct
visual assessment of how well our residuals match what we would expect
from a normal distribution. Outliers, skew, heavy and light-tailed
aspects of distributions (all violations of normality) show up in this
plot once you learn to read it -- which is our next task. To make it
easier to read QQ-plots, it is nice to start with just considering
histograms and/or density plots of the residuals and to see how that
maps into this new display. We can obtain the residuals from the linear
model using the \texttt{residuals} function on any linear model object.




\begin{Shaded}
\begin{Highlighting}[]
\KeywordTok{par}\NormalTok{(}\DataTypeTok{mfrow=}\KeywordTok{c}\NormalTok{(}\DecValTok{1}\NormalTok{,}\DecValTok{2}\NormalTok{))}
\NormalTok{eij<-}\KeywordTok{residuals}\NormalTok{(lm2)}
\KeywordTok{hist}\NormalTok{(eij, }\DataTypeTok{main=}\StringTok{"Histogram of residuals"}\NormalTok{)}
\KeywordTok{plot}\NormalTok{(}\KeywordTok{density}\NormalTok{(eij), }\DataTypeTok{main=}\StringTok{"Density plot of residuals"}\NormalTok{, }\DataTypeTok{ylab=}\StringTok{"Density"}\NormalTok{,}
     \DataTypeTok{xlab=}\StringTok{"Residuals"}\NormalTok{)}
\end{Highlighting}
\end{Shaded}

\begin{figure}
\centering
\includegraphics{GreenwoodBanner_files/figure-latex/Figure3-10-1.pdf}
\caption{\label{fig:Figure3-10}Histogram and density curve of the linear model raw
residuals.}
\end{figure}

Figure \ref{fig:Figure3-10} shows that there is a right skew present in
the residuals for the prisoner rating data model that accounted for
different means in the three groups, which is consistent with the
initial assessment of some right skew in the plots of observations in
each group.

A Quantile-Quantile plot (\textbf{\emph{QQ-plot}}) shows the ``match''
of an observed distribution with a theoretical distribution, almost
always the normal distribution. They are also known as Quantile
Comparison, Normal Probability, or Normal Q-Q plots, with the last two
names being specific to comparing results to a normal distribution. In
this version\footnote{Along with multiple names, there is variation of
  what is plotted on the x and y axes and the scaling of the values
  plotted, increasing the challenge of interpreting QQ-plots. We are
  consistent about the x and y axis choices but different functions that
  make these plots in R do switch the axes.}, the QQ-plots display the
value of observed percentiles in the residual distribution on the y-axis
versus the percentiles of a theoretical normal distribution on the
x-axis. If the observed \textbf{distribution of the residuals matches
the shape of the normal distribution, then the plotted points should
follow a 1-1 relationship.} If the points follow the displayed straight
line then that suggests that the residuals have a similar shape to a
normal distribution. Some variation is expected around the line and some
patterns of deviation are worse than others for our models, so you need
to go beyond saying ``it does not match a normal distribution''. It is
best to be specific about the type of deviation you are detecting. And
to do that, we need to practice interpreting some QQ-plots.

The QQ-plot of the linear model residuals from Figure
\ref{fig:Figure3-9} is extracted and enhanced it a little to make Figure
\ref{fig:Figure3-11} so we can just focus on it. We know from looking at
the histogram that this is a slightly right skewed distribution. The
QQ-plot places the observed \textbf{\emph{standardized}}\footnote{Here
  this means re-scaled so that they should have similar scaling to a
  standard normal with mean 0 and standard deviation 1. This does not
  change the shape of the distribution but can make outlier
  identification simpler -- having a standardized residual more extreme
  than 5 or -5 would suggest a deviation from normality since we rarely
  see values that many standard deviations from the mean in a normal
  distribution. But mainly focus on the shape of the pattern in the
  QQ-plot.} \textbf{\emph{residuals}} on the y-axis and the theoretical
normal values on the x-axis. The most noticeable deviation from the 1-1
line is in the lower left corner of the plot. These are for the negative
residuals (left tail) and there are many residuals at around the same
value that are a little smaller than -1. If the distribution had
followed the normal distribution here, the points would be on the 1-1
line and there would be some standardized residuals much smaller than
-1.5. So we are not getting as much spread in the smaller residuals as
we would expect in a normal distribution. If you go back to the
histogram you can see that the smallest residuals are all stacked up and
do not spread out like the left tail of a normal distribution should. In
the right tail (positive) residuals, there is also a systematic lifting
from the 1-1 line to larger values in the residuals than the normal
would generate. For example, the point labeled as ``82'' (the 82nd
observation in the data set) has a value of 3 in residuals but should
actually be smaller (maybe 2.5) if the distribution was normal. Put
together, this pattern in the QQ-plot suggests that the left tail is too
compacted (too short) and the right tail is too spread out -- this is
the right skew we identified from the histogram and density curve!



\includegraphics{GreenwoodBanner_files/figure-latex/Figure3-11-1.pdf}
\includegraphics{GreenwoodBanner_files/figure-latex/Figure3-11-2.pdf}

Generally, when both tails deviate on the same side of the line (forming
a sort of quadratic curve, especially in more extreme cases), that is
evidence of a skew. To see some different potential shapes in QQ-plots,
six different data sets are displayed in Figures \ref{fig:Figure3-12}
and \ref{fig:Figure3-13}. In each row, a QQ-plot and associated density
curve are displayed. If the points are both above the 1-1 line in the
lower and upper tails as in Figure \ref{fig:Figure3-12}(a), then the
pattern is a right skew, here even more extreme than in the real data
set. If the points are below the 1-1 line in both tails as in Figure
\ref{fig:Figure3-12}(c), then the pattern is identified as a left skew.
Skewed residual distributions (either direction) are problematic for
models that assume normally distributed responses but not necessarily
for our permutation approaches if all the groups have similar skewed
shapes. The other problematic pattern is to have more spread than a
normal curve as in Figure \ref{fig:Figure3-12}(e) and (f). This shows up
with the points being below the line in the left tail (more extreme
negative than expected by the normal) and the points being above the
line for the right tail (more extreme positive than the normal
predicts). We call these distributions \textbf{\emph{heavy-tailed}}
which can manifest as distributions with outliers in both tails or just
a bit more spread out than a normal distribution. Heavy-tailed residual
distributions can be problematic for our models as the variation is
greater than what the normal distribution can account for and our
methods might under-estimate the variability in the results. The
opposite pattern with the left tail above the line and the right tail
below the line suggests less spread (\textbf{\emph{lighter-tailed}})
than a normal as in Figure \ref{fig:Figure3-12}(g) and (h). This pattern
is relatively harmless and you can proceed with methods that assume
normality safely as they will just be a little conservative.




\begin{figure}
\centering
\includegraphics{GreenwoodBanner_files/figure-latex/Figure3-12-1.pdf}
\caption{\label{fig:Figure3-12}QQ-plots and density curves of four simulated
distributions with different shapes.}
\end{figure}

Finally, to help you calibrate expectations for data that are actually
normally distributed, two data sets simulated from normal distributions
are displayed in Figure \ref{fig:Figure3-13}. Note how neither follows
the line exactly but that the overall pattern matches fairly well.
\textbf{You have to allow for some variation from the line in real data
sets} and focus on when there are really noticeable issues in the
distribution of the residuals such as those displayed above. Again, you
will never be able to prove that you have normally distributed residuals
even if the residuals are all exactly on the line, but if you see
QQ-plots as in Figure \ref{fig:Figure3-12} you can encounter situations
that provide evidence of clear violations of the normality assumption.




\begin{figure}
\centering
\includegraphics{GreenwoodBanner_files/figure-latex/Figure3-13-1.pdf}
\caption{\label{fig:Figure3-13}Two more simulated data sets, generated from normal
distributions.}
\end{figure}

The last issues with assessing the assumptions in an ANOVA relates to
situations where the methods are more or less
\textbf{\emph{resistant}}\footnote{A resistant procedure is one that is
  not severely impacted by a particular violation of an assumption. For
  example, the median is resistant to the impact of an outlier.} to
violations of assumptions. For reasons beyond the scope of this book,
the parametric ANOVA F-test is more resistant to violations of the
assumptions of the normality and equal variance assumptions if the
design is balanced. A \textbf{\emph{balanced design}} occurs when each
group is measured the same number of times. The resistance decreases as
the data set becomes less balanced, as the sample sizes in the groups
are more different, so having close to balance is preferred to a more
imbalanced situation if there is a choice available. There is some
intuition available here -- it makes some sense that you would have
better results in comparing groups if the information available is
similar in all the groups and none are relatively under-represented. We
can check the number of observations in each group to see if they are
equal or similar using the \texttt{tally} function from the
\texttt{mosaic} package. This function is useful for being able to get
counts of observations, especially for cross-classifying observations on
two variables that is used in Chapter \ref{chapter5}. For just a single
variable, we use \texttt{tally(\textasciitilde{}x,\ data=...)}:

\begin{Shaded}
\begin{Highlighting}[]
\KeywordTok{require}\NormalTok{(mosaic)}
\KeywordTok{tally}\NormalTok{(}\OperatorTok{~}\NormalTok{Attr, }\DataTypeTok{data=}\NormalTok{MockJury)}
\end{Highlighting}
\end{Shaded}

\begin{verbatim}
## Attr
##    Beautiful      Average Unattractive 
##           39           38           37
\end{verbatim}

So the sample sizes do vary among the groups and the design is
technically not balanced, but it is also very close to being balanced
with only two more observations in the largest group compared to the
smallest group size. This tells us that the \(F\)-test should have some
resistance to violations of assumptions. This nearly balanced design,
and the moderate sample size (over 37 per group is considered a good but
not large sample), make the parametric and nonparametric approaches
provide similar results in this data set even in the presence of the
skewed residual error distribution.

\section{Guinea pig tooth growth One-Way ANOVA
example}\label{section3-5}

A second example of the One-way ANOVA methods involves a study of length
of odontoblasts (cells that are responsible for tooth growth) in 60
Guinea Pigs (measured in microns) from Crampton (1947). \(N=60\) Guinea
Pigs were obtained from a local breeder and each received one of three
dosages (0.5, 1, or 2 mg/day) of Vitamin C via one of two delivery
methods, Orange Juice (\emph{OJ}) or ascorbic acid (the stuff in vitamin
C capsules, called \(VC\) below) as the source of Vitamin C in their
diets. Each guinea pig was randomly assigned to receive one of the six
different treatment combinations possible (OJ at 0.5 mg, OJ at 1 mg, OJ
at 2 mg, VC at 0.5 mg, VC at 1 mg, and VC at 2 mg). The animals were
treated similarly otherwise and we can assume lived in separate cages
and only one observation was taken for each guinea pig, so we can assume
the observations are independent. We need to create a variable that
combines the levels of delivery type (OJ, VC) and the dosages (0.5, 1,
and 2) to use our One-Way ANOVA on the six levels. The
\texttt{interaction} function can be used create a new variable that is
based on combinations of the levels of other variables. Here a new
variable is created in the \texttt{ToothGrowth} data.frame that we
called \texttt{Treat} that provides a six-level grouping variable for
our One-Way ANOVA to compare the combinations of treatments. To get a
sense of the pattern of observations in the data set, the counts in
\texttt{supp} (supplement type) and \texttt{dose} are provided.

\begin{Shaded}
\begin{Highlighting}[]
\KeywordTok{data}\NormalTok{(ToothGrowth) }\CommentTok{#Available in Base R}
\KeywordTok{require}\NormalTok{(mosaic)}
\KeywordTok{tally}\NormalTok{(}\OperatorTok{~}\NormalTok{supp,}\DataTypeTok{data=}\NormalTok{ToothGrowth) }\CommentTok{#Supplement Type (VC or OJ)}
\end{Highlighting}
\end{Shaded}

\begin{verbatim}
## supp
## OJ VC 
## 30 30
\end{verbatim}

\begin{Shaded}
\begin{Highlighting}[]
\KeywordTok{tally}\NormalTok{(}\OperatorTok{~}\NormalTok{dose,}\DataTypeTok{data=}\NormalTok{ToothGrowth) }\CommentTok{#Dosage level}
\end{Highlighting}
\end{Shaded}

\begin{verbatim}
## dose
## 0.5   1   2 
##  20  20  20
\end{verbatim}

\begin{Shaded}
\begin{Highlighting}[]
\CommentTok{#Creates a new variable Treat with 6 levels}
\NormalTok{ToothGrowth}\OperatorTok{$}\NormalTok{Treat=}\KeywordTok{with}\NormalTok{(ToothGrowth,}\KeywordTok{interaction}\NormalTok{(supp,dose)) }

\CommentTok{#New variable that combines supplement type and dosage}
\KeywordTok{tally}\NormalTok{(}\OperatorTok{~}\NormalTok{Treat,}\DataTypeTok{data=}\NormalTok{ToothGrowth) }
\end{Highlighting}
\end{Shaded}

\begin{verbatim}
## Treat
## OJ.0.5 VC.0.5   OJ.1   VC.1   OJ.2   VC.2 
##     10     10     10     10     10     10
\end{verbatim}

The \texttt{tally} function helps us to check for balance; this is a
balanced design because the same number of guinea pigs
(\(n_j=10 \text{ for } j=1, 2,\ldots, 6\)) were measured in each
treatment combination.

With the variable \texttt{Treat} prepared, the first task is to
visualize the results using boxplots and beanplots\footnote{Note that to
  see all the group labels in the plot when making the figure, you have
  to widen the plot window before copying the figure out of R. You can
  resize the plot window using the small vertical and horizontal ``=''
  signs in the grey bars that separate the different panels in RStudio.}
(Figure \ref{fig:Figure3-14}) and generate some summary statistics for
each group using \texttt{favstats}.




\begin{Shaded}
\begin{Highlighting}[]
\KeywordTok{par}\NormalTok{(}\DataTypeTok{mfrow=}\KeywordTok{c}\NormalTok{(}\DecValTok{1}\NormalTok{,}\DecValTok{2}\NormalTok{))}
\KeywordTok{boxplot}\NormalTok{(len}\OperatorTok{~}\NormalTok{Treat,}\DataTypeTok{data=}\NormalTok{ToothGrowth,}\DataTypeTok{ylab=}\StringTok{"Tooth Growth in microns"}\NormalTok{)}
\KeywordTok{beanplot}\NormalTok{(len}\OperatorTok{~}\NormalTok{Treat,}\DataTypeTok{data=}\NormalTok{ToothGrowth,}\DataTypeTok{log=}\StringTok{""}\NormalTok{,}\DataTypeTok{col=}\StringTok{"yellow"}\NormalTok{,}
         \DataTypeTok{method=}\StringTok{"jitter"}\NormalTok{,}\DataTypeTok{ylab=}\StringTok{"Tooth Growth in microns"}\NormalTok{)}
\end{Highlighting}
\end{Shaded}

\begin{figure}
\centering
\includegraphics{GreenwoodBanner_files/figure-latex/Figure3-14-1.pdf}
\caption{\label{fig:Figure3-14}Boxplot and beanplot of tooth growth responses for the six
treatment level combinations.}
\end{figure}

\begin{Shaded}
\begin{Highlighting}[]
\KeywordTok{favstats}\NormalTok{(len}\OperatorTok{~}\NormalTok{Treat,}\DataTypeTok{data=}\NormalTok{ToothGrowth)}
\end{Highlighting}
\end{Shaded}

\begin{verbatim}
##    Treat  min     Q1 median     Q3  max  mean       sd  n missing
## 1 OJ.0.5  8.2  9.700  12.25 16.175 21.5 13.23 4.459709 10       0
## 2 VC.0.5  4.2  5.950   7.15 10.900 11.5  7.98 2.746634 10       0
## 3   OJ.1 14.5 20.300  23.45 25.650 27.3 22.70 3.910953 10       0
## 4   VC.1 13.6 15.275  16.50 17.300 22.5 16.77 2.515309 10       0
## 5   OJ.2 22.4 24.575  25.95 27.075 30.9 26.06 2.655058 10       0
## 6   VC.2 18.5 23.375  25.95 28.800 33.9 26.14 4.797731 10       0
\end{verbatim}

Figure \ref{fig:Figure3-14} suggests that the mean tooth growth
increases with the dosage level and that OJ might lead to higher growth
rates than VC except at a dosage of 2 mg/day. The variability around the
means looks to be small relative to the differences among the means, so
we should expect a small p-value from our \(F\)-test. The design is
balanced as noted above (\(n_j=10\) for all six groups) so the methods
are some what resistant to impacts from non-normality and non-constant
variance. There is some suggestion of non-constant variance in the plots
but this will be explored further below when we can remove the
difference in the means and combine all the residuals together. There
might be some skew in the responses in some of the groups but there are
only 10 observations per group so skew in the boxplots could be
generated by impacts of very few of the observations.

Now we can apply our 6+ steps for performing a hypothesis test with
these observations. The initial step is deciding on the claim to be
assessed and the test statistic to use. This is a six group situation
with a quantitative response, identifying it as a One-Way ANOVA where we
want to test a null hypothesis that all the groups have the same
population mean, at least to start. We will use a 5\% significance
level.

\begin{enumerate}
\def\labelenumi{\arabic{enumi}.}
\item
  \textbf{Hypotheses}:
  \(\boldsymbol{H_0: \mu_{OJ0.5} = \mu_{VC0.5} = \mu_{OJ1} = \mu_{VC1} = \mu_{OJ2} = \mu_{VC2}} \textbf{ vs } \boldsymbol{H_A:} \textbf{ Not all } \boldsymbol{\mu_j} \textbf{ equal}\)

  \begin{itemize}
  \item
    The null hypothesis could also be written in reference-coding as

\begin{verbatim}
$$\boldsymbol{H_0: \tau_{VC0.5} = \tau_{OJ1} = \tau_{VC1} = \tau_{OJ2} = 
\end{verbatim}

    \tau\_\{VC2\}=0\}\$\$
  \end{itemize}
\end{enumerate}

\section{Multiple (pair-wise) comparisons using Tukey's HSD and the
compact letter display}\label{section3-6}

\section{Pair-wise comparisons for Prisoner Rating
data}\label{section3-7}

\section{Chapter Summary}\label{section3-8}

\section{Summary of important R code}\label{section3-9}

\section{Practice problems}\label{section3-10}

\chapter{Two-Way ANOVA}\label{chapter4}

\chapter{Chi-square tests}\label{chapter5}

\chapter{Correlation and Simple Linear Regression}\label{chapter6}

\chapter{Simple linear regression inference}\label{chapter7}

\chapter{Multiple linear regression}\label{chapter8}

\chapter{Case studies}\label{chapter9}

\bibliography{packages,book}


\end{document}
