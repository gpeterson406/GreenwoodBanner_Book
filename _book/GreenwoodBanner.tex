\documentclass[]{book}
\usepackage{lmodern}
\usepackage{amssymb,amsmath}
\usepackage{ifxetex,ifluatex}
\usepackage{fixltx2e} % provides \textsubscript
\ifnum 0\ifxetex 1\fi\ifluatex 1\fi=0 % if pdftex
  \usepackage[T1]{fontenc}
  \usepackage[utf8]{inputenc}
\else % if luatex or xelatex
  \ifxetex
    \usepackage{mathspec}
  \else
    \usepackage{fontspec}
  \fi
  \defaultfontfeatures{Ligatures=TeX,Scale=MatchLowercase}
\fi
% use upquote if available, for straight quotes in verbatim environments
\IfFileExists{upquote.sty}{\usepackage{upquote}}{}
% use microtype if available
\IfFileExists{microtype.sty}{%
\usepackage{microtype}
\UseMicrotypeSet[protrusion]{basicmath} % disable protrusion for tt fonts
}{}
\usepackage[margin=1in]{geometry}
\usepackage{hyperref}
\hypersetup{unicode=true,
            pdftitle={A Second Semester Statistics Course with R},
            pdfauthor={Mark Greenwood and Katherine Banner},
            pdfborder={0 0 0},
            breaklinks=true}
\urlstyle{same}  % don't use monospace font for urls
\usepackage{natbib}
\bibliographystyle{apalike}
\usepackage{longtable,booktabs}
\usepackage{graphicx,grffile}
\makeatletter
\def\maxwidth{\ifdim\Gin@nat@width>\linewidth\linewidth\else\Gin@nat@width\fi}
\def\maxheight{\ifdim\Gin@nat@height>\textheight\textheight\else\Gin@nat@height\fi}
\makeatother
% Scale images if necessary, so that they will not overflow the page
% margins by default, and it is still possible to overwrite the defaults
% using explicit options in \includegraphics[width, height, ...]{}
\setkeys{Gin}{width=\maxwidth,height=\maxheight,keepaspectratio}
\IfFileExists{parskip.sty}{%
\usepackage{parskip}
}{% else
\setlength{\parindent}{0pt}
\setlength{\parskip}{6pt plus 2pt minus 1pt}
}
\setlength{\emergencystretch}{3em}  % prevent overfull lines
\providecommand{\tightlist}{%
  \setlength{\itemsep}{0pt}\setlength{\parskip}{0pt}}
\setcounter{secnumdepth}{5}
% Redefines (sub)paragraphs to behave more like sections
\ifx\paragraph\undefined\else
\let\oldparagraph\paragraph
\renewcommand{\paragraph}[1]{\oldparagraph{#1}\mbox{}}
\fi
\ifx\subparagraph\undefined\else
\let\oldsubparagraph\subparagraph
\renewcommand{\subparagraph}[1]{\oldsubparagraph{#1}\mbox{}}
\fi

%%% Use protect on footnotes to avoid problems with footnotes in titles
\let\rmarkdownfootnote\footnote%
\def\footnote{\protect\rmarkdownfootnote}

%%% Change title format to be more compact
\usepackage{titling}

% Create subtitle command for use in maketitle
\newcommand{\subtitle}[1]{
  \posttitle{
    \begin{center}\large#1\end{center}
    }
}

\setlength{\droptitle}{-2em}
  \title{A Second Semester Statistics Course with R}
  \pretitle{\vspace{\droptitle}\centering\huge}
  \posttitle{\par}
  \author{Mark Greenwood and Katherine Banner}
  \preauthor{\centering\large\emph}
  \postauthor{\par}
  \predate{\centering\large\emph}
  \postdate{\par}
  \date{2017-05-26}

\usepackage{booktabs}

\begin{document}
\maketitle

{
\setcounter{tocdepth}{1}
\tableofcontents
}
\chapter*{Acknowledgments}\label{acknowledgments}
\addcontentsline{toc}{chapter}{Acknowledgments}

We would like to thank all the students and instructors who have
provided input in the development of the current version of STAT 217 and
that have impacted the choice of topics and how we try to teach them.
Dr.~Robison-Cox initially developed this course using R and much of this
work retains his initial ideas. Many years of teaching these topics and
helping researchers use these topics has helped to refine how they are
presented here. Observing students years after the course has also
helped to refine what we try to teach in the course, trying to prepare
these students for the next levels of statistics courses that they might
encounter and the next class where they might need or want to use
statistics.

I (Greenwood) have intentionally taken a first person perspective at
times to be able to include stories from some of those interactions to
try to help you avoid some of their pitfalls in your current or future
usage of statistics. I would like to thank my wife, Teresa Greenwood,
for allowing me the time and support to work on this. I would also like
to acknowledge Dr.~Gordon Bril (Luther College) who introduced me to
statistics while I was an undergraduate and Dr.~Snehalata Huzurbazar
(University of Wyoming) that guided me to completing my Master's and
Ph.D.~in Statistics and still serves as a valued mentor and friend to
me.

The development of this text was initially supported with funding from
Montana State University's Instructional Innovation Grant Program with a
grant titled Towards more active learning in STAT 217. This book was
born with the goal of having a targeted presentation of topics that we
cover (and few that we don't) that minimizes cost to students and
incorporates the statistical software R from day one and every day after
that. The software is a free, open-source platform and so is dynamically
changing over time. This has necessitated frequent revisions of the
text.

This is Version 3.01 of the book. It fixes a problem created with the
digital links in the book that occurred during Spring 2017. Version 3.0
of the book, prepared for Fall 2016, involved edits, a couple of
partially new sections, and updated R code along with a new format for
how the R code is displayed to more easily distinguish it from other
text. Each revision has involved a similar amount of change with Version
2.0 published in January 2015 and Version 1.0 in January 2014 after
using draft chapters that were initially developed during Fall 2013.

We have made every attempt to keep costs as low as possible by making it
possible for most pages to be printed in black and white. The text (in
full color and with dynamic links) is also available as a free digital
download from Montana State University's ScholarWorks repository at
\url{https://scholarworks.montana.edu/xmlui/handle/1/2999}.

Enjoy your journey from introductory to intermediate statistics!

This work is licensed under the Creative Commons
Attribution-NonCommercial-NoDerivatives 4.0 International License. To
view a copy of this license, visit
\url{http://creativecommons.org/licenses/by-nc-nd/4.0/} or send a letter
to Creative Commons, 444 Castro Street, Suite 900, Mountain View,
California, 94041, USA.

Trying to render \(\alpha\)

\chapter{Placeholder}\label{placeholder}

\bibliography{packages,book}


\end{document}
